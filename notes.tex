\section{Non-uniform Nyquist}
A selection of quotations showing approaches to a non-uniform Nyquist limit:

\begin{description}

\item[Average Sampling Rate]

\citet{Scargle82} (citing the following: Beutler 1966, 1970; Masry \& Lui 1975; Higgins 1976; Wiley 1978; Gaster \& Roberts 1975, 1977; Kar, Hornkohl \& Farmer 1981; Ludeman 1981 \todo{read and summarize these!}):
\begin{quote}
Error-free recovery of a band-limited signal [i.e.~reproduction of the entire function $X(t)$ from the samples $X(t_i)$] can be achieved with irregular sampling as long as the mean sampling rate exceeds the Nyquist rate (i.e. the average number of samples per unit time must exceed twice the highest frequency component in the signal).
\end{quote}

\citet{NumRec}: 
\begin{quote}
One guide to choosing $f_{hi}$ is to compare it with the Nyquist frequency $f_c$ which would obtain if the $N$ data points were evenly spaced over the same span $T$, that is $f_c = N/(2T)$. The accompanying program includes an input parameter {\tt hifac}, defined as $f_{hi}/f_{c}$.
\end{quote}

\citet{Horne86}:
\begin{quote}
The largest frequency we calculated was $\pi N/T$ which
is the traditional Nyquist frequency for evenly-spaced data. The Nyquist
frequency is not well defined for unevenly spaced signals, but it can serve as
a reasonable upper limit for the calculation.
\end{quote}

\item[Harmonic Average of Sampling Rate]

\citet{Debosscher07}:
\begin{quote}
For the highest frequency, we used the average of the inverse time intervals
between the measurements: $f_N = 0.5(1/\Delta T)$ as a pseudo
Nyquist frequency. Note that $f_N$ is equal to the Nyquist frequency
in the case of equidistant sampling. For particular cases,
an even higher upper limit can be used \citep[see][]{Eyer99}.
Our upper limit should be seen as a compromise between
the required resolution to allow a good fitting, and computation
time.
\end{quote}


\item[Minimum Sample Spacing]

\citet{Percy86}:
\begin{quote}
... the Nyquist frequency is not well defined for input data obtained at unequally spaced time intervals, the usual case in astronomy \citep{Scargle82}.
Theoretically such a data set contains informatin on the periodicities down to
$\Delta t = \min(t_i - t{i-1})$. In practice, however, a pseudo-Nyquist frequency may be defined by averaging $\Delta t = (t_i - t_{i-1})$, where large, uncharacteristic temporal gaps are avoided. Alternatively, the harmonic mean of all $\Delta t$ may be used. The result is that a useful pseudo-Nyquist frequency may be defined by $\nu_{Ny} = (2 \langle\Delta t \rangle)^{-1}$.
\end{quote}

\citet{Press89}:
\begin{quote}
It is often meaningful to examine frequencies significantly higher than the
Nyquist frequency that would obtain if the same number of data points were
evenly spaced in the same total length of time. Some spectral information is
obtainable for frequencies all the way up to something like half the inverse
spacing of the {\it closest} spaced points.
\end{quote}

\citet{Roberts87}:
\begin{quote}
...for arbitrary $\{t_r\}$, the sampling theorem tells us nothing.
If the data samples are otherwise equally spaced but with missing
points, the theorem says that the data completely determine
a function whose FT is zero for $|v| > 1/(2\Delta_{max})$, where $\Delta_{max}$
is the {\it largest completely sampled} data spacing. However,
there are smaller spacings, and these certainly carry information
about frequencies greater than $1/(2\Delta_{max} )$; some information
is available about frequencies as high as
$1/(2\Delta_{min} )$, where $\Delta_{min}$ is the smallest spacing between data
points. Furthermore, if the $\{t_r\}$ are more or less randomly
distributed, so that a wide range ofspacings are present and
there is little redundancy in the spacing between various
points, tests have shown (Paper II) that significant information
is available on frequencies greater than $l/(2\Delta_{min})$.

Nonetheless, in the present paper we will restrict ourselves
to frequencies obeying
$\nu < \nu_{max} = 1/(2\Delta_{min})$.
\end{quote}

\citet{Hilditch01}:
\begin{quote}
The highest frequency, for equally spaced data, is formally given by the {\it Nyquist frequency}, $f_N = 1/(2\Delta t)$, where $\Delta t$ is the {\it sampling interval} in the data string.
For unequally spaced data, it is common practice to estimate a pseudo-Nyquist frequency from the minimum value of $\Delta t$.
\end{quote}

\item[Getting it Right]

\citet{ICVG2014}:
\begin{quote}
As a good choice for the maximum search frequency, a pseudo-Nyquist frequency $\omega_{max} = \pi/\Delta t$, where $1/\Delta t$ is the median of the inverse time interval between data points, was proposed by \citet{Debosscher07} (in the case of even sampling, $\omega_{max}$ is equal to the Nyquist frequency). In practice, this choice may be a gross underestimate because unevenly sampled data can detect periodicity with frequencies even higher than $2\pi / \Delta t_{min}$ \citep[see][]{Eyer99}. An appropriate choice of $\omega_{max}$ thus depends on sampling (the phase coverage at a given frequency is the relevant quantity) and needs to be carefully chosen: a hard limit on maximum detectable frequency is of course given by the time interval over which individual measurements are performed, such as imaging exposure time.
\end{quote}

\end{description}
